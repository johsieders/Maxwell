\documentclass{article}

\usepackage[english]{babel}
\usepackage[a4paper,top=2cm,bottom=2cm,left=3cm,right=3cm,marginparwidth=1.75cm]{geometry}

\usepackage{amsmath}
\usepackage{amsthm}
\usepackage{amsfonts}
\usepackage{graphicx}
\usepackage[colorlinks=true, allcolors=blue]{hyperref}
\newtheorem{theorem}{Theorem}[section]
\newtheorem{corollary}{Corollary}[theorem]
\newtheorem{lemma}[theorem]{Lemma}

\DeclareMathOperator\supp{supp}
\DeclareMathOperator\vol{vol}


\title{Vector Calculus Explained}
\author{Johannes Siedersleben}
\date{December 2022}
\setlength\parindent{0pt}

\begin{document}
\maketitle

\begin{abstract}
This paper explains the essentials of vector calculus.
\end{abstract}

\section{Introduction}

Your introduction goes here!

$a = \supp{f}$

$v = \vol{C}$

$v = \det{C}$

\section{Prerequisites}
Our aim is to derive the essentials of vector calculus in a straightforward way.


\section{Transformation of Variables}

This presentation is inspired by \cite{Forster3}.


We are going to investigate how integrals transform if we replace a variable $x$ with $g(x)$. We are going to proceed in three steps.



\begin{theorem}
 Let $C$ be the unit cube of $\mathbb{R}^{n}$, $A \in \mathbb{R}^{n\times n}$ an invertible matrix and 
 $g : \mathbb{R}^{n} \longrightarrow \mathbb{R}^n$ a differentiable function whose Jacobian $\partial g$ is everywhere invertible. Then the following equations hold:

\begin{equation} \label{eq: volume}
\lvert A(C) \rvert =  \lvert \det A  \rvert \cdot \lvert C \rvert
\end{equation}

\begin{equation} \label{eq: linear_subst}
 \int f(y) dy = \int (f\circ A) (x) \cdot  \lvert \det A \rvert dx
\end{equation}

\begin{equation} \label{eq: gen_subst}
 \int f(y) dy = \int (f\circ g) (x) \cdot  \lvert \det \partial g  \rvert dx
\end{equation}

\begin{equation} \label{eq: gen_subst_short}
 y = g(x) \implies dy = \lvert \det \partial g \rvert dx
\end{equation}
\end{theorem}

Equation \ref{eq: volume} tells us how a given volume grows or shrinks on a large scale, and equation \ref{eq: gen_subst_short} how infinitesimal volume elements behave.


\begin{proof}
We start with equation \ref{eq: volume}. If $A$ is diagonal 

\begin{equation*}
    A = \left[
    \begin{array}{ccc}
        \lambda_{1}&   \\
            & \ddots  \\
            && \lambda_{n}
    \end{array}
\right]
\end{equation*}

then $A(C)$ is a cuboid with side lengths $\lambda_1, \hdots, \lambda_n$, and volume $\lvert A(C)\rvert = \prod_{i = 1}^{n} \lvert \lambda_i \rvert = \lvert \det A \rvert$. Note that only the absolute values of the eigenvalues contribute to the volume; the signs (indicating the quadrant) are irrelevant.
If $A$ is unitary (a rotation), it leaves lengths, angles, and hence volumes unchanged. If $A$ is diagonalizable, then we have $A = U^T \Lambda U$ with a unitary matrix $U$ and the diagonal matrix $\Lambda$ of eigenvalues. Applying $A$ comes down to applying $U$ (volume unchanged), $\Lambda$ (volume multiplied by $\lvert \det \Lambda \rvert$) and $U^T$ (volume unchanged). So, equation \ref{eq: volume} is proven for diagonalizable matrices. Now, the matrix $A^TA$ is always diagonalizable, and its eigenvalues are positive. We will come up with two different unitary matrices $U$ and $V$ and a diagonal matrix $M$ such that

\begin{equation} \label{eq: umv}
A = UMV^T
\end{equation}
and 
\begin{equation} \label{eq: detA}
\det(A) = \det(M)
\end{equation}

Equation \ref{eq: detA} follows from \ref{eq: umv}, and we are done if the latter is shown. Now, there is a unitary matrix $V$ such that:

\begin{equation} \label{eq:Lambda}
\Lambda = V^{T}(A^TA)V
\end{equation}

We define two diagonal matrices

\begin{equation*}
\begin{array}{cc}
\Lambda = \left[
    \begin{array}{ccc}
        \lambda_{1}&   \\
            & \ddots  \\
            && \lambda_{n}
    \end{array}
\right]
& M = \left[
    \begin{array}{ccc}
        \sqrt{\lambda_{1}}&   \\
            & \ddots  \\
            && \sqrt{\lambda_{n}}
    \end{array}
\right]
\end{array}
\end{equation*}

with the square roots in $M$ understood to be positive. It clearly holds that:

\begin{equation*}
M^2 = \Lambda
\end{equation*}

and, multiplying twice by $M^{-1}$:

\begin{equation} \label{eq:Mlambda}
I = M^{-1}\Lambda M^{-1}
\end{equation}

Plugging equation \ref{eq:Lambda} into \ref{eq:Mlambda} allows us to define another unitary matrix $U$:

\begin{equation*}
I = \underbrace{M^{-1} V^{T}A^T}_{U^T}\underbrace{AV M^{-1}}_{U}
\end{equation*}

and we get equation \ref{eq: umv}, the desired result:

\begin{equation*}
A = AVM^{-1}M V^T = UMV^T
\end{equation*}

We now turn to equation \ref{eq: linear_subst}. Let
$ \mathcal{V} = \{(V_i, y_i) | i = 1, \hdots, N\}$
be a partition of $\supp{f}$. Then $\mathcal{W} = \{(A^{-1}(V_i), A^{-1}y_i) | i = 1, \hdots, N\}$ is a partition of $A^{-1}(\supp{f})$.We set $W_i = A^{-1}(V_i)$ and $x_i = A^{-1}y_i $ so $V_i = A(W_i)$ and $y_i = Ax_i$. We get:


\begin{equation*}
\begin{split}
\int f(y) dy &\approx \sum_{i = 1}^{N} f(y_i) \lvert V_i \rvert \\ 
             &= \sum_{i = 1}^{N} f(Ax_i) \lvert A(W_i) \rvert \\
             &= \sum_{i = 1}^{N} f(Ax_i) \lvert \det{A} \rvert \lvert W_i \rvert \\
             &\approx \int f(Ax) \lvert \det{A} \rvert dx
\end{split}              
\end{equation*}

\end{proof}

Let's test equation \ref{eq: volume} on the simplest possible example: The matrix  
$A = \begin{bmatrix}
a_1 & b_1 \\
a_2 & b_2
\end{bmatrix}$
maps the unit cube, in $\mathbb{R}^2$ a square with side length $ = 1$, to a parallelogram with surface
$F = \lVert a \rVert \cdot \lVert b \rVert \cdot \sin\phi$,
where $
a = \begin{bmatrix}
a_1 \\
a_2
\end{bmatrix}$,
$b = \begin{bmatrix}
b_1 \\
b_2
\end{bmatrix}$
and $\phi$ is given by $\cos\phi = \frac{\lVert a \rVert \cdot \lVert b}{a \cdot b}$. We expect $F = \lvert \det A \rvert$, which turns out to be true: 


\begin{equation*}
\begin{split}
F &= \lVert a \rVert \cdot \lVert b \rVert \cdot \sqrt{1 - \cos^2 \phi} \\
  &= \sqrt{\lVert a \rVert \cdot \lVert b  \rVert - (a \cdot b)^2} \\
  &= \sqrt{a_1^2 b_2^2 - 2 a_1a_2 b_1 b_2 + a_2^2 b_1^2} \\
  &= \lvert a_1 b_2 - a_2 b_1 \rvert \\
  &= \lvert \det A \rvert
\end{split}
\end{equation*}


Let us now look at the polar coordinates in two dimensions. Here they are, with $r, \phi$ instead of $x_1, x_2$:

\begin{align*}
y &= g_2(r, \phi) \\
\begin{bmatrix}
y_1 \\
y_2
\end{bmatrix} &= \begin{bmatrix}
r \cos \phi \\
r \sin \phi
\end{bmatrix} \\
\partial g_2 &= \begin{bmatrix}
\partial y_1 / \partial r &\partial y_1 / \partial \phi \\
\partial y_2 / \partial r &\partial y_2 / \partial \phi \\
\end{bmatrix} = \begin{bmatrix}
\cos \phi & -r  \sin \phi \\
\sin \phi & r  \cos \phi \\
\end{bmatrix} \\
\det \partial g_2 &= r
\end{align*}


The transformation $g_2$ maps the rectangle $[0, R] \times [0, \pi]$ onto the hemicycle $H_{R}$, so we get the surface

\begin{equation*}
\lvert H_{R} \rvert = \int_{H_{R}} dy =  \int_0^{R} (r \int_0^{\pi} d\phi) dr = \int_0^{R} r \pi dr = \frac{\pi}{2} R^2
\end{equation*}

The integrand $r \pi$ is the circumference of $H_{r}$, half the circumference of the full cycle.

The same works in three dimensions: 
\begin{align*}
y &=  g_3(r, \phi, \theta) \\
\begin{bmatrix}
y_1 \\
y_2 \\
y_3
\end{bmatrix} &= \begin{bmatrix}
r \sin \phi \cos \theta \\
r \sin \phi \sin \theta \\
r \cos \phi 
\end{bmatrix} \\
\partial g_3 &= \begin{bmatrix}
\sin \phi \: \cos \theta & r  \cos \phi \: \cos \theta & -r \sin \phi \: \sin \theta \\
\sin \phi \: \sin \theta & r  \cos \phi \: \sin \theta & r \sin \phi \: \cos \theta \\
\cos \phi             & -r  \sin \phi            & 0
\end{bmatrix} \\
\det \partial g_3 &= r^2 \sin \phi
\end{align*}

The transformation $g_3$ maps the cuboid $[0, R] \times [0, \pi] \times [0, 2 \pi]$ onto the hemisphere $S_{R}$, so we get the volume

\begin{equation*}
\lvert S_{R} \rvert = \int_{S_{R}} dy =  \int_0^{R} (r^2 \int_0^{2\pi} \int_0^{\pi} \sin \phi \: d\phi d\theta) dr = \int_0^R 2 \pi r^2 dr = \frac{2 \pi}{3} R^3
\end{equation*}

The integrand $2 \pi r^2$ is the surface of  $S_{r}$, half the surface of the full sphere.

\subsection{Euler-Lagrange}

some stuff goes here


\bibliographystyle{alpha}
\bibliography{calculus}

\end{document}